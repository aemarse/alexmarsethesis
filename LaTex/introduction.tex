% introduction.tex
% 
% LaTeX document from NYU music technology master's thesis

\section{Introduction}
\label{introduction}

%%%%%%%%%%%%%%%%%%%%%%%%%%%%%%%%%%%%%%%

This is an introductory chapter.\footnote{This is a footnote}  More text goes here.


\subsection{This is a subsection}
Contra quos omnis dicendum breviter existimo. Quamquam philosophiae quidem vituperatoribus satis responsum est eo libro, quo a nobis philosophia defensa et collaudata est, cum esset accusata et vituperata ab Hortensio. qui liber cum et tibi probatus videretur et iis, quos ego posse iudicare arbitrarer, plura suscepi veritus ne movere hominum studia viderer, retinere non posse. 

\subsubsection{This is a subsubsection}
Sive enim ad sapientiam perveniri potest, non paranda nobis solum ea, sed fruenda etiam [sapientia] est; sive hoc difficile est, tamen nec modus est ullus investigandi veri, nisi inveneris, et quaerendi defatigatio turpis est, cum id, quod quaeritur, sit pulcherrimum. etenim si delectamur, cum scribimus, quis est tam invidus, qui ab eo nos abducat? 

\subsubsection{This is a subsubsection}
Iis igitur est difficilius satis facere, qui se Latina scripta dicunt contemnere. 

\subsection{This is another subsection}
Quamquam, si plane sic verterem Platonem aut Aristotelem, ut verterunt nostri poetae fabulas, male, credo, mererer de meis civibus, si ad eorum cognitionem divina illa ingenia transferrem. sed id neque feci adhuc nec mihi tamen, ne faciam, interdictum puto. locos quidem quosdam, si videbitur, transferam, et maxime ab iis, quos modo nominavi, cum inciderit, ut id apte fieri possit, ut ab Homero Ennius, Afranius a Menandro solet. 

\subsubsection{This is a subsubsection}
Ego autem quem timeam lectorem, cum ad te ne Graecis quidem cedentem in philosophia audeam scribere? quamquam a te ipso id quidem facio provocatus gratissimo mihi libro, quem ad me de virtute misisti. 

\paragraph{This is a ``paragraph"}
Ego autem quem timeam lectorem, cum ad te ne Graecis quidem cedentem in philosophia audeam scribere? quamquam a te ipso id quidem facio provocatus gratissimo mihi libro, quem ad me de virtute misisti. 

\subparagraph{This is a ``subparagraph"}
Ego autem quem timeam lectorem, cum ad te ne Graecis quidem cedentem in philosophia audeam scribere? quamquam a te ipso id quidem facio provocatus gratissimo mihi libro, quem ad me de virtute misisti. 
