% methodology.tex
% 
% LaTeX document from NYU music technology master's thesis


\section{Methodology}
\label{methodology}

%%%%%%%%%%%%%%%%%%%%%%%%%%%%%%%%%%%%%%%

\subsection{Now let's have some equations}

Here is a simple equation:

\begin{equation}
\delta(x,y) = i + j + k 
\end{equation}

\noindent where $\delta(x,y) =$ the distance between chord $x$ and chord $y$,  $i =$ the number of steps between two regions on the chromatic fifths circle (i.e. distance between two chords with regard to key), $j =$ the number of steps between two chords on the diatonic fifths circle (distance with regard to chord function), and $k =$ the number of distinctive pitch classes in the basic space of $y$ compared to those in the basic space of $x$. 


And then we have another equation,

\begin{equation}
\alpha(p_1,p_2) = \bigg{(}\frac{s_2}{s_1}\bigg{)}\bigg{(}\frac{1}{n^2}\bigg{)}
\end{equation}

\noindent where $\alpha(p_1,p_2) =$ the melodic attraction of pitch $p_1$ to $p_2$, $s_1 =$ the anchoring strength of $p_1$, $s_2 =$ anchoring strength of $p_2$ in the current configuration of the basic space, and $n =$ the number of semitone intervals between $p_1$ and $p_2$.


And finally, 

\begin{equation}
T_{global} = \delta(x,y) + T_{diss} + \alpha(p_1,p_2) + t
\end{equation}



\noindent where $t$ is the inherited tension value derived from a prolongational reduction.


