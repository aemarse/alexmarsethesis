% analysis.tex
% 
% LaTeX document from NYU music technology master's thesis


\section{Analysis}
\label{analysis}

%%%%%%%%%%%%%%%%%%%%%%%%%%%%%%%%%%%%%%%

\subsection{Let's create a table}
See below for Table \ref{testtable}.
\begin{table*}[!ht]

\begin{center}
\begin{footnotesize}
\vspace*{.25in}
\begin{tabular}{| c | l |}
\hline
Stability rating  &  Key and chord context\\ 
\hline
\hline  
6 & \parbox[t]{.7\textwidth}{Chord root (and, after a seventh in the melody, the pitch one diatonic step down from it} \\
\hline
5 & \parbox[t]{.7\textwidth}{Chord third and fifth} \\
\hline
4 & \parbox[t]{.7\textwidth}{Other diatonic pitches} \\
\hline
2 & \parbox[t]{.7\textwidth}{Chord root (Chromatic pitches} \\
\hline

\end{tabular}
\caption{\footnotesize Table showing stability ratings and the requirements which govern them.}
\label{testtable}
\end{footnotesize}
\end{center}
\end{table*}


